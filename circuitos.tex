\documentclass{article}
\usepackage[american]{circuitikz}
\usepackage{caption}
\usepackage[spanish]{babel}

\begin{document}
% \begin{circuitikz} \draw
%   (0,0) to [#1,#2] (2,0) ;
% \end{circuitikz}
% #1: nombre del componente
% #2: opciones separadas por coma (opcional)
\begin{figure}
  \begin{circuitikz} \draw
    (0,0) to [R] (2,0) ;
  \end{circuitikz}
    \caption{Resistencia pelada}
\end{figure}


\begin{figure}
  \begin{circuitikz} \draw
    (0,0) to [R=$R_1$] (2,0) ;
  \end{circuitikz}
  \caption{agrego el nombre}
\end{figure}

\begin{figure}
  \begin{circuitikz} \draw
    (0,0) to [R=$R_1$,i=$i_1$] (2,0) ;
  \end{circuitikz}
  \caption{agrego la corriente}
\end{figure}

\begin{figure}
  \begin{circuitikz} \draw
    (0,0) to [R=$R_1$,i=$i_1$,v=$v_1$] (2,0) ;
  \end{circuitikz}
  \caption{agrego la tension}
\end{figure}


\begin{figure}
  \begin{circuitikz} \draw
    (0,0) to [V=$V$] (0,2)--
    (0,2) to [R=$R_1$] (2,2)--
    (2,2) to [R=$R_2$] (2,0)--(0,0);
  \end{circuitikz}
  \caption{divisor de tension}
\end{figure}

\begin{figure}
  \begin{circuitikz} \draw
    (0,0) to [sV=$V$] (0,4)--
    (0,4) to [R=$R_1$] (4,4)--
    (4,4) to [L=$L_1$] (4,2)--
    (4,2) to [C=$C_1$] (4,0)--(0,0)
    ;
  \end{circuitikz}
  \caption{RLC serie}
\end{figure}









\end{document}
